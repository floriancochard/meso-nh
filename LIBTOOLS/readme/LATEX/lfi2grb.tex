\section{Conversion to GRIB or Vis5D files}

\subsection{Presentation}
FM synchronous file can be convert into \underline{GRIB} 
or \underline{Vis5D} format.
This section aims at describ how the converter works and how use it.

The GRIB (GRId in Binary) format is a standard meteorological one, defined 
by the WMO. GRIB files can be plotted with METVIEW
%\footnote{available on {\tt xdata} workstation in CNRM}
graphic interface (developped at ECMWF), or 
R2\footnote{used in the GMME/MICADO team at CNRM} software.

The Vis5D format is specified for using Vis5D\footnote{home page 
{\tt http://www.ssec.wisc.edu/\~ billh/vis5d.html}}
software (following the GNU General Public License): 3 spatial
dimensions, time dimension, 5$^{th}$ dimension for enumeration of variables.
It is rather designed for animation of 3D plotting.

Choice was made to put together the two file formats in a same conversion
program because in both cases specificities of Meso-NH grids have to be
treated in the same way (horizontally: Arakawa C-grid, vertically: Gal-Chen
coordinate $\hat z$ following terrain). However, the user has to choose one 
of the two formats available when running the tool
(see section \ref{s:execution}).


\subsection{Usage} \label{s:execution}
The interactive tool is
called {\tt lfi2grb} or {\tt lfi2v5d} according the wanted output
file format, but it runs the same program. Some questions are to be
answered to indicate the number and type of vertical levels, the type of
horizontal domain,
and the name of the variables to write into the output file.
All that is typed on keyboard is saved in {\tt dirconv.grb} or {\tt dirconv.v5d}
file, it can be appended and used as input (after renaming it) for the next call
of the tool (e.g. {\tt mv dirconv.grb dirgrb ; lfi2grb < dirgrb}).

For historical reasons, a program with the same goal of conversion to GRIB or 
Vis5d has been first developped as a main program
of MesoNH, as DIAG program is. This program called {\bf CONVLFI} runs with
the MesoNH procedure {\bf prepmodel} and
a namelist file {\tt CONVLFI1.nam} (see \ref{ss:convlfi}). 

To use the converter after a {\bf DIAG prepmodel} job, the Meso-NH file must
remain a synchronous file, not transformed onto a diachronic file:
in {\bf prepmodelrc} specify {\tt OUTFILE\_TOOLS='fm'} 
(default is 'conv2dia' to convert with {\tt conv2dia}).


\subsubsection{{\tt lfi2grb} tool}
When {\tt lfi2grb} tool is invoked, you must indicate, 
after the name of the input file, first 
the horizontal grid (type, eventually type of interpolation and domain),
the vertical grid (type and levels), 
then the list of the 3-dimensional fields to convert, 
and the list of the 2-dimensional ones.

For the \underline{horizontal grid}, you can either keep the one of MesoNH file
(cartesien or conformal projection) or interpolate onto a lat-lon regular grid.
In the first case, you can replace all the fields on mass points (A-grid)
or keep the native grid (C-grid).
In the second case, you have to indicate
the bounds of the domain with north and south latitudes and west and east
longitudes, as well as the type of horizontal interpolation:
nearest-neighbour value or bilinear interpolation with the 4 surrounding values.
The resolution of the lat.-lon. grid is automatically initialized 
with the equivalent value of the grid-mesh where the map scale is minimum. 
The program also indicates the number of grid points of the Meso-NH domain 
inside the prescribed lat-lon domain. If there are points of lat-lon domain
outside Meso-NH one, the value of the interpolated fields at these points
will be a missing one.

The \underline{vertical grid} can be either the native K levels or pressure
levels. 
In the first case ({\tt K}), all levels are kept and no interpolation is done:
the height specified in the GRIB header is the one of the grid without orography.
In the second case ({\tt P}), the list of pressure levels is either specified
manually or computed using a linear function from user-specified
minimum, maximum and increment values. If a prescribed level is below the lower
Meso-NH level or above the upper MesoNH level, the value of the field at this 
level will be a missing one. Otherwise, the value is computed from
a linear interpolation in log(P).

The \underline{3-dimensional fields} to convert are specified as follows: 
one field per line with first the name of the record in the input file
following by its grib code (tabular character is allowed). Note that no test
is done on the value of grib code (GRIB header {\sf ISEC1(6)}): you choose it
to easily identify the field with the software used after the conversion.
The end of the list is indicated by the keyword {\tt END}.

The \underline{2-dimensional fields} to convert are specified as follows:
one field per line with first the name of the record in the input file
(it can be a K-level of a 3-dimensional field too),
following by its grib code and possibly level indicator and level value
(tabular character is allowed).
Note that the value of the level indicator ({\sf ISEC1(7)}) is optional 
(the default value is 105: {\sf 'specified height above ground'}).
So is the level value ({\sf ISEC1(8)}), the default value is the altitude of 
the first mass point of the K-levels.
The end of the list is indicated by the keyword {\tt END}.

\subsubsection{Example of {\tt lfi2grb} use}
\begin{itemize}
\item to convert onto a GRIB file with horizontal and vertical interpolations in P levels:\\
(all that is typed on keyboard (in {\it italic} in the example below)
is saved in {\tt dirconv.grb})
\end{itemize}
\small
{\tt - ENTER FM synchronous FILE NAME (without .lfi) ?}  \\
{\tt\it CEXP.1.CSEG.001d } \hspace{3.5cm} $<$- the input file must be splitted in .des and .lfi \\
{\tt - Horizontal interpolation to lat-lon regular grid? (Y/y/O/o/N/n)}\\
{\tt\it y }  \\
{\tt - Type of interpolation? NEARest-neighbour (default) or BILInear }\\
{\tt\it NEAR }  \\
{\tt  - NSWE target domain bounds (in degrees)? }\\
{\tt\it 55. 35. -20. 10. }  \\
{\tt - Vertical grid: type K or P ? }\\
{\tt\it P }  \\
{\tt - Type of vertical grid: given by linear FUNCTN (default) or MANUALly ?}\\
{\tt\it FUNCTN }  \\
{\tt - Enter number of P levels ?} \\
{\tt\it 5 }  \\
{\tt - Values of the  5  P levels (hPa, from bottom to top):} \\
{\tt\it 1000. 850. 700. 500. 300. }  \\
{\tt - Enter 3D variables to CONVERT (1/1 line, end by END): }\\
{\tt  MesoNH field name, grib parameter indicator }\\
{\tt\it UM  33 }\\
{\tt - next 3D field or END ? }\\
{\tt\it VM  34 }\\
{\tt - next 3D field or END ? }\\
{\tt\it END }\\
{\tt - Enter 2D variables to CONVERT (1/1 line, end by END): }\\
{\tt  MesoNH field name, grib parameter indicator, eventually level indicator and level value}\\
{\tt\it T2M  13  105  2}\\
{\tt - next 2D field or END ? }\\
{\tt\it THM\_K\_2  13}\\
{\tt - next 2D field or END ? }\\
{\tt\it END}\\
{\tt 2 fields (3D), and   2 fields (2D) written in CEXP.1.CSEG.001d.GRB }\\
 
\normalsize
\subsubsection{{\tt lfi2v5d} tool}
When {\tt lfi2v5d} tool is invoked, you must indicate, 
after the name of the input file, first 
the vertical grid (type and levels), 
then the list of the 3-dimensional fields to convert, 
and the list of the 2-dimensional ones.

No horizontal interpolation is available for the Vis5D format output: all the
converted fields are replaced on mass points (A-grid) of the MesoNH grid
(cartesien or conformal projection).

The \underline{vertical grid} can be either the native K levels, altitude
levels or pressure levels. 
In the first case ({\tt K}), all levels are kept and the fields are interpolated
on the levels of the lowest point of the domain.
In the second and third cases ({\tt Z} and {\tt P}), the list of levels is
either specified
manually or computed using a linear function from user-specified
minimum, maximum and increment values. The value of the field is computed from
a linear interpolation in Z or in log(P).

The \underline{3-dimensional fields} to convert are specified with 
one record name per line.
The end of the list is indicated by the keyword {\tt END}.

Then the \underline{2-dimensional fields},
or a K-level of 3-dimensional fields,
to convert are specified in the same way.

\subsubsection{Example of {\tt lfi2v5d} use}
\begin{itemize}
\item to convert onto a Vis5D file with vertical interpolation in Z levels:\\
(all that is typed on keyboard (in {\it italic} in the example below)
is saved in {\tt dirconv.v5d})
\end{itemize}
\small
{\tt - ENTER FM synchronous FILE NAME (without .lfi) ?}  \\
{\tt\it CEXP.1.CSEG.001 } \hspace{3.5cm} $<$- the input file must be splitted in .des and .lfi \\
{\tt - Verbosity level ?}  \\
{\tt\it 5 }  \\
{\tt - File 2D (xz): L2D=T or F ?}  \\
{\tt\it F }  \\
{\tt - Vertical grid: type K,Z or P ?}  \\
{\tt\it Z }  \\
{\tt - Type of vertical grid: given by linear FUNCTN (default) or MANUALly ?} \\
{\tt\it FUNCTN }  \\
{\tt - Vertical grid: min, max, int (m for Z, hPa for P)?} \\
{\tt\it 1500 9000 3000 }  \\
{\tt - Enter 3D variables to CONVERT (1/1 line, end by END): }\\
{\tt\it THM }  \\
{\tt - next 3D field or END ? }\\
{\tt\it POVOM }  \\
{\tt - next 3D field or END ? }\\
{\tt\it END }\\
{\tt - Enter 2D variables to CONVERT (1/1 line, end by END): }\\
{\tt\it ZS }  \\
{\tt - next 2D field or END ? }\\
{\tt\it END }\\
{\tt 2 fields (3D), and   1 fields (2D) written in CEXP.1.CSEG.001d.V5D }\\

\subsubsection{{\bf CONVLFI} program} \label{ss:convlfi}
The MesoNH program {\bf CONVLFI} allows conversion onto GRIB 
(the horizontal grid is either the native 
MesoNH grid (Arakawa C-grid) of the field, the MesoNH mass grid
(Arakawa A-grid),
the vertical grid is either the native K levels or pressure levels), or
conversion onto Vis5D (the horizontal grid is the MesoNH mass grid
(A-grid), the vertical grid is either the native K levels without orography,
altitude or pressure levels).

The conversion is done with the Meso--NH procedure {\bf prepmodel} used with 
the {\bf CONVLFI} program and the {\tt CONVLFI1.nam} namelist file.
Up to 24 FM files can be treated identically in a single prepmodel job.
\\

A) In the file \underline{\bf prepmodelrc}, the input and output host, directories
and login control variables refer to the input and output files as usual.
The other control variables to initialize specifically in this file are:
\begin{itemize}
\item MAINPROG=CONVLFI
\item LOAD\_OPT='location\_of\_v5d\_library'
\item OUTHOST=name\_workstation  (for example) \\
this allows future use of {\tt vis5d} or {\tt metview} on your local host.
\end{itemize}

B) In the \underline{\tt CONVLFI1.nam} namelist file, the user must indicate
the format type wanted, the number and type of vertical levels, 
the type of horizontal interpolation on a lat/lon domain 
as well as the name of the variables to write into the output file:
\begin{enumerate}
\item\underline{Namelist NAM\_OUTFILE}: 

\begin{center}
\begin{tabular} {|l|l|l|}
\hline
Fortran name & Fortran type & default value\\
\hline
\hline
CMNHFILE     & array of character (len=28)  & none   \\
COUTFILETYPE & character (len=3)   & none   \\
NVERB        & integer    & 5  \\
LAGRID       & logical    & .TRUE.  \\
CLEVTYPE     & character (len=1)   & 'P' if COUTFILETYPE='GRB'   \\
             &                     & 'K' if COUTFILETYPE='V5D'   \\
CLEVLIST     & character (len=6)   & 'FUNCTN'   \\
XVLMIN       & real    & 10000.  if COUTFILETYPE='GRB'  \\
XVLMAX       & real    & 100000. if COUTFILETYPE='GRB'  \\
XVLINT       & real    & 10000.  if COUTFILETYPE='GRB'  \\
LLMULTI      & logical    & .TRUE. \\
\hline
\end{tabular}
\end{center}

\begin{itemize}
\item CMNHFILE: name of the input FM file (from an initialization sequence, or
a model simulation, or after diagnostics computation).
\index{CMNHFILE!\innam{NAM\_OUTFILE}}
\item COUTFILETYPE: type of the output file, appended
to CMNHFILE to generate the name of the output file.
\begin{itemize}
\item 'V5D'
\item 'GRB'
\end{itemize}
\index{COUTFILETYPE!\innam{NAM\_OUTFILE}}
\item NVERB: verbosity level 
\begin{itemize}
\item  0 for minimum of prints
\item  5 for intermediate level of prints
\item  10 for maximum of prints.
\end{itemize}
\index{NVERB!\innam{NAM\_OUTFILE}}
\item LAGRID: switch to interpolate fields on an Arakawa A-grid (mass grid),
\subitem forced to .TRUE. if Vis5D file or horizontal interpolation.
\index{LAGRID!\innam{NAM\_OUTFILE}}
\item CLEVTYPE: type of vertical levels in output file,
\index{CLEVTYPE!\innam{NAM\_OUTVER}}
\begin{itemize}
\item 'P' pressure levels
\item 'Z' z levels (only used for COUTFILETYPE='V5D')
\item 'K' 
\subitem if COUTFILETYPE='GRB': native vertical grid of Meso-NH (no
interpolation, height specified in GRIB message is the one of the grid 
without orography),
\subitem if COUTFILETYPE='V5D': native vertical grid of Meso-NH (fields are
interpolated on the levels of the lowest point of the domain).
\end{itemize}
\item CLEVLIST: how vertical levels are specified
\begin{itemize}
\item 'MANUAL' number and list of levels specified in the 1$^{st}$ free-format
part,
\item 'FUNCTN' using a linear function, with the next 3 parameters. 
\end{itemize}
\index{CLEVLIST!\innam{NAM\_OUTVER}}
\item XVLMIN: minimum value for the vertical grid 
\subitem (in m for CLEVTYPE = 'Z', in Pa for CLEVTYPE = 'P'),
\item XVLMAX: maximum value for the vertical grid (`'),
\item XVLINT: increment value for the vertical grid (`').
\item LLMULTI: switch to produce a multigrib file (.T.) or monogrib files (.F.),
only used for COUTFILETYPE='GRB' (each monogrib file name is composed with the
date, the variable name and the level).
\index{LLMULTI!\innam{NAM\_OUTFILE}}

\end{itemize}

\item\underline{Free-format part}: (number and list of vertical levels) \\
This part is only used if CLEVLIST='MANUAL':
\begin{enumerate}
\item first the number of vertical levels,
\item then the list of levels, by increasing values in m if CLEVTYPE = 'Z', or decreasing
values in Pa if CLEVTYPE = 'P'
\end{enumerate}

\item\underline{Free-format part}: (variable names)
This part indicates the record name of the variables of the input file to
write in the output file. It is specified in two parts:
\begin{enumerate}
\item between the keywords BEGIN\_3D and END\_3D: the name of the 3D fields,
following by their grib code if COUTFILETYPE='GRB' (separed by tabular 
character).
\item between the keywords BEGIN\_2D and END\_2D: the name of the 2D fields,
following by their grib code, and possibly level indicator and level value
if COUTFILETYPE='GRB' (separed by tabular character).
\end{enumerate}
{\bf N.B.:} do not forget the comment line after the keyword BEGIN\_3D
and BEGIN\_2D.


\end{enumerate}

\underline{C) Example of namelist file CONVLFI1.nam}
\begin{itemize}
\item
to convert into a Vis5d file:
\end{itemize}

\begin{verbatim}
&NAM_OUTFILE  CMNHFILE(1)='T1E20.2.09B24.002',
              CMNHFILE(2)='T1E20.2.09B24.003',
              COUTFILETYPE='V5D',
              CLEVTYPE='Z', CLEVLIST='MANUAL',
              LAGRID=T, NVERB=10 /
15
30.
100.
250.
500.
1000.
1500.
2000.
2500.
3000.
3500.
4000.
4500.
5000.
6000.
8000.

BEGIN_3D
#variables 3D (MesoNH field name)
UM
VM
WM
THM
END_3D
BEGIN_2D
#variables 2D (MesoNH field name)
ZS
END_2D
\end{verbatim}

\begin{itemize}
\item
to convert into a GRIB file:
\end{itemize}
\begin{verbatim}
&NAM_OUTFILE  CMNHFILE(1)='T1E20.2.09B24.002',
              CMNHFILE(2)='T1E20.2.09B24.003',
              COUTFILETYPE='GRB', 
              CLEVTYPE='P', CLEVLIST='FUNCTN',
              XVLMAX=100000., XVLMIN=10000., XVLINT=10000.,
              LAGRID=T, NVERB=5 /

BEGIN_3D
#variables 3D (MesoNH field name, grib parameter indicator)
UM  33
VM  34
WM  40
THM 13
END_3D
BEGIN_2D
#variables 2D (MesoNH field name, grib parameter indicator)
ZS 8
END_2D
next lines are ignored
codes example:
MSLP    1  
ACPRR   61 
INPRR   59 
PABSM   1
ALT 6
TEMP    11
REHU    52
RVM 53
RCM 153
RRM 170
RIM 178
RSM 171
RGM 179
RHM 226
RARE    230
HHRE    231
VVRE    232
VDOP    233
POVOM   234
\end{verbatim}


\normalsize
\subsection{Short description of the program}
Two main tasks are performed by the program: 
\begin{enumerate}
 \item \subitem After the specification of the name of the input file, a `light'
initialization subroutine {\tt init\_for\_convlfi.f90 } is called to initialize
the I/O interface, the geometry, dimensions, grids, metric coefficients, times,
and to read pressure field.
 \subitem According the output grids choosen, extra arrays are allocated for 
interpolations. 
\ignore{
If horizontal interpolation is required, the equivalent 
resolution and the number of usefull points are computed by the subroutine 
{\tt ini2lalo.f90}.
}%ignore
 \item Then fields are treated one after another: first 3D fields, then 
2D fields. 
 \subitem In the case of GRIB conversion, fields are interpolated and written
one after another (subroutine {\tt code\_and\_write\_grib.f90 } called for each
horizontal level of each field). 
 \subitem For Vis5D conversion, fields are interpolated and written
all together (subroutine \newline {\tt code\_and\_write\_vis5d.f90 } called at the end).
\end{enumerate}
Using a `light' initialization routine and reading fields name from standard
input allows the conversion program not to be dependant of a MesoNH version 
or program.


\subsection{Some tips to use Vis5D}
See the complete guide for using Vis5D: file README.ps in the Vis5D package.

\subsubsection{Utilities} (section 5 of README.ps)
\begin{itemize}
\item
{\tt v5dinfo filename}: shows summary of the v5d file: number and name of
the variables, size of the 3-D grid, number of time steps, vertical
grid definition and projection definition.
\item
{\tt v5dstats filename}: shows statistics of the v5d file:
minimum value, maximum value, mean value, standard deviation of
 each variable.
\item
{\tt v5dedit filename}: edits the header of the v5d file and allows to change
it: variables names, variables units, times and dates, projection, vertical
coordinate system, low levels. \\
{\it Useful to set the variable's units since they are not set by the program
 CONVLFI.}
\item
{\tt v5dappend [-var] filename1 ... targetfile}: joins v5d files together: 
{\it useful since the {\bf prepmodel} job generates a separate v5d file for each
 timestep}, {\tt var} indicates list of variables to omit in the target file,
the dimensions of 3-D grids must be the same in each input file.
\end{itemize}

\subsubsection{Options} \label{ss:opt} (section 6.1 of README.ps) \\

To call Vis5D: {\tt vis5d file1 [options] file2 [options] ...} \\
Options can be be specified here when calling, or by pressing the {\sf DISPLAY}
button of the main control panel and then the 'Options' menu.

Options useful to set when calling: \\
{\tt [-date]} use 'dd month yy' instead of julian 'yyddd' date, \\
{\tt [-box x y z]} specify the aspect ratio of the 3-D box (default is 2 2 1), \\
{\tt [-mbs n]} override the assumed system memory size of 32 megabytes (Vis5D
tells you value to specify if not enough), \\
{\tt [-topo file]} use a topography file other than the default EARTH.TOPO


\subsubsection{Control panel} (section 6.2 of README.ps) \\
The top buttons control primary functions of Vis5D (see section
\ref{sss:funct}). \\
The middle ones control the viewing modes (see section \ref{sss:viewing}).\\
The bottom 2-D matrix of buttons contains physical variables on the rows, and
types of graphic representation on the columns. To control any type of graphic,
click on the button with the left mouse button. 
A pop-up window appears when clicking with the middle mouse button, and
one window to modify colors with the right button  
(see section \ref{sss:graph}).
\\

\underline{\bf Primary functions} \label{sss:funct}(section 6.3 of README.ps)
\begin{itemize}
\item{\sf SAVE PIC} to save the image in a file: first toggle the {\sf REVERSE}
button to reverse black and white, then toggle the {\sf SAVE PIC} button and
choose {\tt xwd} (X Window Dump) format. The file can be visualised with
 {\tt xv} utility and transformed into {\tt postscript} format.

\item{\sf GRID\#s} to display the grid indices instead of latitude, longitude and
vertical units along the edges of the box.

\item{\sf CONT\#s, LEGENDS} to toggle on or off the isoline values, the colorbar
legends.

\item{\sf BOX, CLOCK} to toggle on or off the display of the box and the clock.

\item{\sf TOP, SOUTH, WEST} to set a top (or bottom), a south (or north), a west
(or east) view.
{\it Select} {\sf SOUTH} {\it to visualise 2D file.}

\item{\sf SAVE, RESTORE, SCRIPT} to save and restore isolines, colors, labels,
view (write and read a Tcl script).

\item{\sf UVW VARS} to specify the names of the variables to use to display wind
slices and trajectories, several triplets of variables can be used.

\item{\sf NEW VAR..} to duplicate variables or create new ones by specifying
mathematical expressions (formulas use names of existing variables, numbers,
arithmetic operations, functions such as $SQRT,EXP,LOG,SIN,COS,TAN,ABS,MIN,MAX$,
ex: horizontal wind speed, $spd=SQRT(UM*UM+VM*VM)$
see section 6.13 of README.ps).

\item{\sf ANIMATE} when several time steps: left mouse button: forward,
right button: backward, S key: slower, F key: faster.

\item{\sf STEP} when several time steps: left mouse button: one step ahead,
 middle button: first step, right button: one step back.

\item{\sf DISPLAY} to change the number of displays, the display options
(see section \ref{ss:opt}), the display parameters (as with the {\tt v5dedit}
utility).

\end{itemize}

\underline{\bf Viewing modes} \label{sss:viewing}(section 6.4 of README.ps) \\
The underlined modes are the most useful (the others are much better displayed
with {\tt diaprog} Meso-NH graphics).
\begin{itemize}
\item\underline{\sf Normal} 
 to rotate, zoom and translate the graphics in the 3D window.

%\item{\sf Trajectory}
% to create and display wind trajectories.
%
\item\underline{\sf Slice}
 to reposition horizontal and vertical slices.

\item\underline{\sf Label}
 to create and edit text labels in the 3D window.

\item{\sf Probe}
 to inspect individual grid values with a cursor moving through the 3D grid.

\item{\sf Sounding}
 to display a vertical sounding at the location of the moveable cursor.

\item{\sf Clipping}
 to reposition the six bounding planes of the 3-D box. Select one plane (top, bottom,
 north, south, west or east) with the middle mouse button, and reposition it
 with the right mouse button.

\end{itemize}

\underline{\bf Types of graphic representations} \label{sss:graph}(sections 6.5 to 6.9 of README.ps) \\
The underlined types are the most useful (the others are much better displayed
with {\tt diaprog} Meso-NH graphics).
\begin{itemize}
\item\underline{\sf Isosurfaces}: 
 A 3-D contour surface showing the volume bounding by a particular value of the
field (set with the left mouse button). The isosurface is either monocolor
or colored according to the values of another variable (right mouse button). 

\item\underline{\sf Slices}: 
Planar cross section (horizontally or vertically) can be moved in this mode.
To replace geographic coordinates by grid
coordinates, press the {\sf "GRID \#s"} button on the control panel.

\subitem contour line: interval can be changed
and min/max values specified in the pop-up window. {\tt -10 (-30,20)} will
plot values between -30 and 20 at intervals 10 with negative values dashed.
Color can be changed with the right mouse button.

\subitem colored slice: colors can be changed in the pop-up window 
(with the mouse buttons or arrow keys). Color table is displayed in the
3-D window if the {\sf "LEGEND \#s"} button is selected. 
%Transparency can be changed by pressing the SHIFT key while using mouse.
To change limits of plotted values, use the keyboard array buttons when in 
the variable control panel (left and right for limits in the extend of the 
variable values, up and down for colors inside it).

\subitem wind vector slice: (buttons {\sf Hwind1, Vwind1, Hwind2, Vwind2})
the scale parameter multiplies the length of vectors drawn
(double: 2, half: 0.5), the density parameter controls the number of vectors
(between zero and one, 0.5 for one vector of two, 0.25 for one of four).

\subitem wind stream slice: (buttons {\sf HStream, VStream})
the density parameter controls the number of streamlines
(between zero and two).

\item\underline{\sf Volume rendering}: {\it for powerful workstations..}
 
\end{itemize}


\subsubsection{Advanced use}

\begin{itemize}
\item generate your own topography file, with the {\tt maketopo.c} program
in the {\tt util} directory (see 5 of README.ps).

\item Tcl language, to write script (button {\sf SCRIPT}) or
interactively (button {\sf INTERP..}) (see 6.16 of README.ps).

\item external analysis functions written in Fortran,
in {\tt userfuncs} directory (see 6.13.3 of README.ps).

\end{itemize}

\subsection{State of art}
The converter only runs on Linux and VPP.
In HP, right compilation options have to be found to use the external library...
