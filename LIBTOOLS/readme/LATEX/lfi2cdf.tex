\section{Conversion to NetCDF files}

\subsection{{\tt lfi2cdf} tool}

The \texttt{lfi2cdf} tool converts the binary part (or LFI file) of a
FM file (synchronous or diachronic) into a NetCDF file.  All the fields
(or more precisely all the LFI articles) contained in the input LFI file
are copied to the NetCDF output file with their values unchanged. As
a LFI article does not hold any information on the variable, the tool
tries to describe the corresponding NetCDF variable by using~:

\begin{itemize}
\item 3 LFI articles: \texttt{IMAX, JMAX,} and \texttt{KMAX}
  if they are available in the LFI input file. These articles may
  provide the NetCDF dimensions \texttt{DIMX, DIMY,} and \texttt{DIMZ}
  of an array variable. If these variables are not available in the
  input file, the tool treats each array variable as a 1D array.
\item a small database implemented as a structure array in the
  \texttt{lfi2cdf} source file \texttt{fieldtype.f90}. This array
  holds the type (\texttt{REAL, INTEGER, LOGICAL}\ldots) of every
  common LFI article. When an article is not present in this database,
  its name is displayed on \texttt{stdout} by the running tool, and
  the corresponding values are always considered as \texttt{REAL}
  values. A new LFI article type description can be easily added in
  the \texttt{fieldtype.f90} source file and the tool must be then
  recompiled.
\end{itemize}

\subsubsection{Usage}
The binary part of the FM file is required in the current directory.
The following commands convert a file \texttt{myfile.lfi} from LFI to NetCDF:

\begin{verbatim}
lfi2cdf myfile.lfi
\end{verbatim} 
or
\begin{verbatim}
lfi2cdf myfile
\end{verbatim}

\noindent The output NetCDF file is named:
\texttt{myfile.cdf}. 
%myfile{\bf .cdf}. 
It can easily be manipulated by NetCDF tools\footnote{see
freely available NetCDF software at  http://www.unidata.ucar.edu/packages/netcdf/software.html} like
\texttt{ncdump}, \texttt{ncview}, or \texttt{NCO} operators.\\

\noindent In the same way, you will convert a NetCDF
file \texttt{myfile.cdf} back to LFI format by typing:

\begin{verbatim}
cdf2lfi myfile.cdf
\end{verbatim}
or
\begin{verbatim}
cdf2lfi myfile
\end{verbatim}
The output LFI file is then named: \texttt{myfile.lfi}


\subsection{{\tt extractdia} tool}
The \texttt{extractdia} tool converts a diachronic FM file into a NetCDF file after an extraction of a list of fields and an optional extraction of a sub-domain.  See the section \ref{extractdia}.


%%% Local Variables: 
%%% mode: latex
%%% TeX-master: "tools"
%%% End: 
