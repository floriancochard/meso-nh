\section{Compression of FM files} 

A specific compression tool has been developed for FM files.  This
tool, called {\tt lfiz}, was first devoted for files that will be
explored by the graphic utility {\tt diaprog}. In fact, it is also
used for files used during a simulation (initial and coupling files)
to reduce the data storage.  Some information of how the compression
works is given here, its execution is particularly easy.

\subsection{{\tt lfiz} tool}

The \texttt{lfiz} tool works on the binary part (LFI file) of a FM
file, synchronous or diachronic.  It is a lossy compression tool.
The compressed articles are exclusively the 2-dimensional or
3-dimensional \texttt{REAL} fields. When dealing with 3D fields the tool works
with each 2D plane on every vertical level. The initial values stored
with 64-bit \texttt{REAL} precision are first converted into 32-bit
\texttt{REAL} precision and then compressed by mapping the 32-bit
real values upon 16-bit integer values (with a possible isolation of
extrema values).  The better compression is
achieved for fields with small value range.  For fields with missing
value (e.g.  2-dimensional fields with land-sea mask), the extremum
value is excluded and the compression is done on significant values of
the field. The minimum compression ratio is 4 for each 2D or 3D
\texttt{REAL} compressed field.

\subsection{{\tt unlfiz} tool}
The \texttt{unlfiz} tool will restore the 64-bit \texttt{REAL} value size to all
the compressed LFI articles. However, each previously compressed article
will gain no more than a 32-bit \texttt{REAL} precision because of the lossy
technique involved above.


\subsection{Usage}
The binary part of the FM file is required in the current
directory. To compress the file \texttt{myfile.lfi}, you can type:

\begin{verbatim}
lfiz myfile.lfi
\end{verbatim}

\noindent This will produce the compressed file \texttt{myfile.Z.lfi}\\


\noindent In the same way, to uncompress the file \texttt{myfile.Z.lfi}, you can
type:
\begin{verbatim}
unlfiz myfile.Z.lfi
\end{verbatim}

\noindent The output file \texttt{myfile.lfi} is a valid LFI file but the LFI
articles previously compressed are 64-bit \texttt{REAL} with no more than 32-bit
\texttt{REAL} precision.




%%% Local Variables: 
%%% mode: latex
%%% TeX-master: "tools"
%%% End: 
